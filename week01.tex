% Options for packages loaded elsewhere
\PassOptionsToPackage{unicode}{hyperref}
\PassOptionsToPackage{hyphens}{url}
\PassOptionsToPackage{dvipsnames,svgnames,x11names}{xcolor}
%
\documentclass[
  letterpaper,
  DIV=11,
  numbers=noendperiod]{scrartcl}

\usepackage{amsmath,amssymb}
\usepackage{iftex}
\ifPDFTeX
  \usepackage[T1]{fontenc}
  \usepackage[utf8]{inputenc}
  \usepackage{textcomp} % provide euro and other symbols
\else % if luatex or xetex
  \usepackage{unicode-math}
  \defaultfontfeatures{Scale=MatchLowercase}
  \defaultfontfeatures[\rmfamily]{Ligatures=TeX,Scale=1}
\fi
\usepackage{lmodern}
\ifPDFTeX\else  
    % xetex/luatex font selection
\fi
% Use upquote if available, for straight quotes in verbatim environments
\IfFileExists{upquote.sty}{\usepackage{upquote}}{}
\IfFileExists{microtype.sty}{% use microtype if available
  \usepackage[]{microtype}
  \UseMicrotypeSet[protrusion]{basicmath} % disable protrusion for tt fonts
}{}
\makeatletter
\@ifundefined{KOMAClassName}{% if non-KOMA class
  \IfFileExists{parskip.sty}{%
    \usepackage{parskip}
  }{% else
    \setlength{\parindent}{0pt}
    \setlength{\parskip}{6pt plus 2pt minus 1pt}}
}{% if KOMA class
  \KOMAoptions{parskip=half}}
\makeatother
\usepackage{xcolor}
\setlength{\emergencystretch}{3em} % prevent overfull lines
\setcounter{secnumdepth}{5}
% Make \paragraph and \subparagraph free-standing
\makeatletter
\ifx\paragraph\undefined\else
  \let\oldparagraph\paragraph
  \renewcommand{\paragraph}{
    \@ifstar
      \xxxParagraphStar
      \xxxParagraphNoStar
  }
  \newcommand{\xxxParagraphStar}[1]{\oldparagraph*{#1}\mbox{}}
  \newcommand{\xxxParagraphNoStar}[1]{\oldparagraph{#1}\mbox{}}
\fi
\ifx\subparagraph\undefined\else
  \let\oldsubparagraph\subparagraph
  \renewcommand{\subparagraph}{
    \@ifstar
      \xxxSubParagraphStar
      \xxxSubParagraphNoStar
  }
  \newcommand{\xxxSubParagraphStar}[1]{\oldsubparagraph*{#1}\mbox{}}
  \newcommand{\xxxSubParagraphNoStar}[1]{\oldsubparagraph{#1}\mbox{}}
\fi
\makeatother

\usepackage{color}
\usepackage{fancyvrb}
\newcommand{\VerbBar}{|}
\newcommand{\VERB}{\Verb[commandchars=\\\{\}]}
\DefineVerbatimEnvironment{Highlighting}{Verbatim}{commandchars=\\\{\}}
% Add ',fontsize=\small' for more characters per line
\usepackage{framed}
\definecolor{shadecolor}{RGB}{241,243,245}
\newenvironment{Shaded}{\begin{snugshade}}{\end{snugshade}}
\newcommand{\AlertTok}[1]{\textcolor[rgb]{0.68,0.00,0.00}{#1}}
\newcommand{\AnnotationTok}[1]{\textcolor[rgb]{0.37,0.37,0.37}{#1}}
\newcommand{\AttributeTok}[1]{\textcolor[rgb]{0.40,0.45,0.13}{#1}}
\newcommand{\BaseNTok}[1]{\textcolor[rgb]{0.68,0.00,0.00}{#1}}
\newcommand{\BuiltInTok}[1]{\textcolor[rgb]{0.00,0.23,0.31}{#1}}
\newcommand{\CharTok}[1]{\textcolor[rgb]{0.13,0.47,0.30}{#1}}
\newcommand{\CommentTok}[1]{\textcolor[rgb]{0.37,0.37,0.37}{#1}}
\newcommand{\CommentVarTok}[1]{\textcolor[rgb]{0.37,0.37,0.37}{\textit{#1}}}
\newcommand{\ConstantTok}[1]{\textcolor[rgb]{0.56,0.35,0.01}{#1}}
\newcommand{\ControlFlowTok}[1]{\textcolor[rgb]{0.00,0.23,0.31}{\textbf{#1}}}
\newcommand{\DataTypeTok}[1]{\textcolor[rgb]{0.68,0.00,0.00}{#1}}
\newcommand{\DecValTok}[1]{\textcolor[rgb]{0.68,0.00,0.00}{#1}}
\newcommand{\DocumentationTok}[1]{\textcolor[rgb]{0.37,0.37,0.37}{\textit{#1}}}
\newcommand{\ErrorTok}[1]{\textcolor[rgb]{0.68,0.00,0.00}{#1}}
\newcommand{\ExtensionTok}[1]{\textcolor[rgb]{0.00,0.23,0.31}{#1}}
\newcommand{\FloatTok}[1]{\textcolor[rgb]{0.68,0.00,0.00}{#1}}
\newcommand{\FunctionTok}[1]{\textcolor[rgb]{0.28,0.35,0.67}{#1}}
\newcommand{\ImportTok}[1]{\textcolor[rgb]{0.00,0.46,0.62}{#1}}
\newcommand{\InformationTok}[1]{\textcolor[rgb]{0.37,0.37,0.37}{#1}}
\newcommand{\KeywordTok}[1]{\textcolor[rgb]{0.00,0.23,0.31}{\textbf{#1}}}
\newcommand{\NormalTok}[1]{\textcolor[rgb]{0.00,0.23,0.31}{#1}}
\newcommand{\OperatorTok}[1]{\textcolor[rgb]{0.37,0.37,0.37}{#1}}
\newcommand{\OtherTok}[1]{\textcolor[rgb]{0.00,0.23,0.31}{#1}}
\newcommand{\PreprocessorTok}[1]{\textcolor[rgb]{0.68,0.00,0.00}{#1}}
\newcommand{\RegionMarkerTok}[1]{\textcolor[rgb]{0.00,0.23,0.31}{#1}}
\newcommand{\SpecialCharTok}[1]{\textcolor[rgb]{0.37,0.37,0.37}{#1}}
\newcommand{\SpecialStringTok}[1]{\textcolor[rgb]{0.13,0.47,0.30}{#1}}
\newcommand{\StringTok}[1]{\textcolor[rgb]{0.13,0.47,0.30}{#1}}
\newcommand{\VariableTok}[1]{\textcolor[rgb]{0.07,0.07,0.07}{#1}}
\newcommand{\VerbatimStringTok}[1]{\textcolor[rgb]{0.13,0.47,0.30}{#1}}
\newcommand{\WarningTok}[1]{\textcolor[rgb]{0.37,0.37,0.37}{\textit{#1}}}

\providecommand{\tightlist}{%
  \setlength{\itemsep}{0pt}\setlength{\parskip}{0pt}}\usepackage{longtable,booktabs,array}
\usepackage{calc} % for calculating minipage widths
% Correct order of tables after \paragraph or \subparagraph
\usepackage{etoolbox}
\makeatletter
\patchcmd\longtable{\par}{\if@noskipsec\mbox{}\fi\par}{}{}
\makeatother
% Allow footnotes in longtable head/foot
\IfFileExists{footnotehyper.sty}{\usepackage{footnotehyper}}{\usepackage{footnote}}
\makesavenoteenv{longtable}
\usepackage{graphicx}
\makeatletter
\newsavebox\pandoc@box
\newcommand*\pandocbounded[1]{% scales image to fit in text height/width
  \sbox\pandoc@box{#1}%
  \Gscale@div\@tempa{\textheight}{\dimexpr\ht\pandoc@box+\dp\pandoc@box\relax}%
  \Gscale@div\@tempb{\linewidth}{\wd\pandoc@box}%
  \ifdim\@tempb\p@<\@tempa\p@\let\@tempa\@tempb\fi% select the smaller of both
  \ifdim\@tempa\p@<\p@\scalebox{\@tempa}{\usebox\pandoc@box}%
  \else\usebox{\pandoc@box}%
  \fi%
}
% Set default figure placement to htbp
\def\fps@figure{htbp}
\makeatother

\KOMAoption{captions}{tableheading}
\makeatletter
\@ifpackageloaded{tcolorbox}{}{\usepackage[skins,breakable]{tcolorbox}}
\@ifpackageloaded{fontawesome5}{}{\usepackage{fontawesome5}}
\definecolor{quarto-callout-color}{HTML}{909090}
\definecolor{quarto-callout-note-color}{HTML}{0758E5}
\definecolor{quarto-callout-important-color}{HTML}{CC1914}
\definecolor{quarto-callout-warning-color}{HTML}{EB9113}
\definecolor{quarto-callout-tip-color}{HTML}{00A047}
\definecolor{quarto-callout-caution-color}{HTML}{FC5300}
\definecolor{quarto-callout-color-frame}{HTML}{acacac}
\definecolor{quarto-callout-note-color-frame}{HTML}{4582ec}
\definecolor{quarto-callout-important-color-frame}{HTML}{d9534f}
\definecolor{quarto-callout-warning-color-frame}{HTML}{f0ad4e}
\definecolor{quarto-callout-tip-color-frame}{HTML}{02b875}
\definecolor{quarto-callout-caution-color-frame}{HTML}{fd7e14}
\makeatother
\makeatletter
\@ifpackageloaded{caption}{}{\usepackage{caption}}
\AtBeginDocument{%
\ifdefined\contentsname
  \renewcommand*\contentsname{Table of contents}
\else
  \newcommand\contentsname{Table of contents}
\fi
\ifdefined\listfigurename
  \renewcommand*\listfigurename{List of Figures}
\else
  \newcommand\listfigurename{List of Figures}
\fi
\ifdefined\listtablename
  \renewcommand*\listtablename{List of Tables}
\else
  \newcommand\listtablename{List of Tables}
\fi
\ifdefined\figurename
  \renewcommand*\figurename{Figure}
\else
  \newcommand\figurename{Figure}
\fi
\ifdefined\tablename
  \renewcommand*\tablename{Table}
\else
  \newcommand\tablename{Table}
\fi
}
\@ifpackageloaded{float}{}{\usepackage{float}}
\floatstyle{ruled}
\@ifundefined{c@chapter}{\newfloat{codelisting}{h}{lop}}{\newfloat{codelisting}{h}{lop}[chapter]}
\floatname{codelisting}{Listing}
\newcommand*\listoflistings{\listof{codelisting}{List of Listings}}
\makeatother
\makeatletter
\makeatother
\makeatletter
\@ifpackageloaded{caption}{}{\usepackage{caption}}
\@ifpackageloaded{subcaption}{}{\usepackage{subcaption}}
\makeatother

\usepackage{bookmark}

\IfFileExists{xurl.sty}{\usepackage{xurl}}{} % add URL line breaks if available
\urlstyle{same} % disable monospaced font for URLs
\hypersetup{
  pdftitle={Classical Linear Regression Model \& Ordinary Least Squares},
  colorlinks=true,
  linkcolor={blue},
  filecolor={Maroon},
  citecolor={Blue},
  urlcolor={Blue},
  pdfcreator={LaTeX via pandoc}}


\title{Classical Linear Regression Model \& Ordinary Least Squares}
\author{}
\date{}

\begin{document}
\maketitle

\renewcommand*\contentsname{Table of contents}
{
\hypersetup{linkcolor=}
\setcounter{tocdepth}{3}
\tableofcontents
}

\begin{tcolorbox}[enhanced jigsaw, title=\textcolor{quarto-callout-note-color}{\faInfo}\hspace{0.5em}{Note}, titlerule=0mm, colbacktitle=quarto-callout-note-color!10!white, left=2mm, opacityback=0, toprule=.15mm, colframe=quarto-callout-note-color-frame, rightrule=.15mm, toptitle=1mm, bottomtitle=1mm, opacitybacktitle=0.6, arc=.35mm, bottomrule=.15mm, leftrule=.75mm, breakable, colback=white, coltitle=black]

\href{week01.pdf}{Download PDF}

\end{tcolorbox}

\section{Two variable linear regression
analysis}\label{two-variable-linear-regression-analysis}

\subsection{Readings}\label{readings}

\begin{itemize}
\tightlist
\item
  Stock and Watson (2003), Chapters 4--5
\item
  Dougherty (2016), Chapter 2
\item
  Wooldridge (2013), Chapter 2
\end{itemize}

\section{Introduction}\label{introduction}

Econometrics means economic measurement. Econometricians attempt to
quantify economic relationships that are of interest to theory and
policy.

\section{Correlation vs Regression}\label{correlation-vs-regression}

\subsection{Correlation}\label{correlation}

In economics we are interested in the relationship between two or more
random variables, for example: - Sales and advertising expenditure -
Personal consumption and disposable income - Investment and interest
rates - Earnings and schooling A measure of linear association between
two random variables \(x\) and \(y\) is the \textbf{covariance}, which
for a sample of \(n\) pairs of observations \((x_1, y_1)\), \(\ldots\),
\((x_n, y_n)\) is calculated as

\[
\operatorname{cov}(x,y)
=
\frac{1}{n-1}
\sum_{i=1}^{n}
(x_i - \bar{x})(y_i - \bar{y})
\]

The covariance is a measure of \textbf{linear} association. It may be
approximately zero even when there is a strong non-linear
(e.g.~quadratic) relationship.

\section{Overview}\label{overview}

In this handout we will revisit the Classical Linear Regression Model
(CLRM) {[}see @wooldridge2010, chap. 1-2{]}. The goal of this week's
lecture is to: 1. understand the model specification; 2. it's underlying
assumptions; 3. and the appropriate interpretation; 4. the OLS
estimator, using linear algebra; 5. the geometry of OLS and partitioned
regression result.

\section{Model Specification}\label{model-specification}

The linear population regression model is given by,

\[ 
\begin{aligned}
  Y_i =& X_i'\beta+\varepsilon_i \\
  =& \beta_1\mathbf{1}+\beta_2X_{i2}+\beta_3X_{i3}+...+\beta_kX_{ik}+\varepsilon_i
\end{aligned}
\]

\begin{Shaded}
\begin{Highlighting}[]
\DecValTok{1}\SpecialCharTok{+}\DecValTok{1}
\end{Highlighting}
\end{Shaded}

\begin{verbatim}
[1] 2
\end{verbatim}

\begin{Shaded}
\begin{Highlighting}[]
\NormalTok{qlfs18 }\OtherTok{\textless{}{-}}\NormalTok{ haven}\SpecialCharTok{::}\FunctionTok{read\_dta}\NormalTok{(}\StringTok{"data/qlfs1314\_2.dta"}\NormalTok{)}
\end{Highlighting}
\end{Shaded}

\begin{Shaded}
\begin{Highlighting}[]
\FunctionTok{library}\NormalTok{(tidyverse)}
\FunctionTok{library}\NormalTok{(haven)}

\CommentTok{\# Read the Stata dataset (adjust path if needed)}
\NormalTok{qlfs18 }\OtherTok{\textless{}{-}} \FunctionTok{read\_dta}\NormalTok{(}\StringTok{"data/qlfs1314\_2.dta"}\NormalTok{)}

\CommentTok{\# Show variables so you can verify names during render}
\FunctionTok{head}\NormalTok{(}\FunctionTok{names}\NormalTok{(qlfs18), }\DecValTok{30}\NormalTok{)}
\end{Highlighting}
\end{Shaded}

\begin{verbatim}
 [1] "age"     "cameyr"  "cry01"   "degcls"  "edage"   "hdpch19" "hhtype" 
 [8] "higho"   "hiqual"  "hourpay" "marsta"  "nsecmmj" "sc2kmmj" "sex"    
[15] "sngdeg"  "soc2kmn" "ttushr"  "uresmc"  "nation"  "eth01"   "relig"  
[22] "hiquala" "hiqualv" "sctqual" "yr"      "tenure"  "hpay"    "lhpay"  
[29] "married" "white"  
\end{verbatim}

\begin{Shaded}
\begin{Highlighting}[]
\CommentTok{\# Rename if those vars exist}
\ControlFlowTok{if}\NormalTok{ (}\StringTok{"HOURPAY"} \SpecialCharTok{\%in\%} \FunctionTok{names}\NormalTok{(qlfs18)) \{}
\NormalTok{  qlfs18 }\OtherTok{\textless{}{-}}\NormalTok{ qlfs18 }\SpecialCharTok{|\textgreater{}} \FunctionTok{rename}\NormalTok{(}\AttributeTok{hourpay =}\NormalTok{ HOURPAY)}
\NormalTok{\}}
\ControlFlowTok{if}\NormalTok{ (}\StringTok{"EDAGE"} \SpecialCharTok{\%in\%} \FunctionTok{names}\NormalTok{(qlfs18)) \{}
\NormalTok{  qlfs18 }\OtherTok{\textless{}{-}}\NormalTok{ qlfs18 }\SpecialCharTok{|\textgreater{}} \FunctionTok{rename}\NormalTok{(}\AttributeTok{edage =}\NormalTok{ EDAGE)}
\NormalTok{\}}
\ControlFlowTok{if}\NormalTok{ (}\StringTok{"AGE"} \SpecialCharTok{\%in\%} \FunctionTok{names}\NormalTok{(qlfs18)) \{}
\NormalTok{  qlfs18 }\OtherTok{\textless{}{-}}\NormalTok{ qlfs18 }\SpecialCharTok{|\textgreater{}} \FunctionTok{rename}\NormalTok{(}\AttributeTok{age =}\NormalTok{ AGE)}
\NormalTok{\}}

\CommentTok{\# Only plot if hourpay exists}
\ControlFlowTok{if}\NormalTok{ (}\StringTok{"hourpay"} \SpecialCharTok{\%in\%} \FunctionTok{names}\NormalTok{(qlfs18)) \{}
  \FunctionTok{ggplot}\NormalTok{(qlfs18, }\FunctionTok{aes}\NormalTok{(}\AttributeTok{x =}\NormalTok{ hourpay)) }\SpecialCharTok{+}
    \FunctionTok{geom\_histogram}\NormalTok{(}\AttributeTok{bins =} \DecValTok{30}\NormalTok{)}
\NormalTok{\} }\ControlFlowTok{else}\NormalTok{ \{}
  \FunctionTok{cat}\NormalTok{(}\StringTok{"hourpay not found — check the variable name in the dataset.}\SpecialCharTok{\textbackslash{}n}\StringTok{"}\NormalTok{)}
\NormalTok{\}}
\end{Highlighting}
\end{Shaded}

\begin{verbatim}
Error in list2(na.rm = na.rm, binwidth = binwidth, bins = bins, orientation = orientation, : object 'ffi_list2' not found
\end{verbatim}

\begin{Shaded}
\begin{Highlighting}[]
\CommentTok{\# Create log wage if possible}
\ControlFlowTok{if}\NormalTok{ (}\StringTok{"hourpay"} \SpecialCharTok{\%in\%} \FunctionTok{names}\NormalTok{(qlfs18)) \{}
\NormalTok{  qlfs18 }\OtherTok{\textless{}{-}}\NormalTok{ qlfs18 }\SpecialCharTok{|\textgreater{}} \FunctionTok{mutate}\NormalTok{(}\AttributeTok{lhourpay =} \FunctionTok{log}\NormalTok{(hourpay))}
  \FunctionTok{summary}\NormalTok{(qlfs18}\SpecialCharTok{$}\NormalTok{lhourpay)}
\NormalTok{\}}
\end{Highlighting}
\end{Shaded}

\begin{verbatim}
   Min. 1st Qu.  Median    Mean 3rd Qu.    Max.    NA's 
 -2.040   2.175   2.526   2.553   2.902   6.880   14658 
\end{verbatim}




\end{document}
